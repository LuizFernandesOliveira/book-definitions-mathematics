\section{Divisor}\label{sec:divisor}
Dados $a, b \in \mathbb{Z}$, dizemos que \textbf{$a$ divide $b$}, e escrevemos $a|b$, se existir $q \in \mathbb{Z}$ tal que $b = qa$.

\section{Primo}\label{sec:primo}
Um número $p \in \mathbb{N}$ e $p \neq 1$, é \textbf{primo} se os únicos números inteiros que o dividem são ele próprio e o 1. Ou seja, $p$ é primo se para todo $b \in \mathbb{N}$ tal que $b|p$ então $b=p$ ou $b=1$.

