\subsubsection*{Máximo Divisor Comum}
Dados $a, b \in \mathbb{Z}$, um número natural $d$ é chamado \textbf{máximo divisor comum} de  $a$ e $b$, denotado por $d \coloneqq mdc(a, b)$, se satisfaz as afirmações abaixo:
    \begin{description}
        \item[(i)] $d|a$ e $d|b$.
        \item[(ii)] se $r \in \mathbb{Z}$, é tal que $r|a$ e $e|b$, então $r|d$.
    \end{description}

\subsubsection*{Múltiplo}
Seja $a \in \mathbb{Z}$. Um número inteiro $b$ é chamado \textbf{múltiplo} de $a$ e $b$ se $b=aq$, para algum $q \in \mathbb{Z}$.
